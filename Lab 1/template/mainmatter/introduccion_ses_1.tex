\chapter{Sesión 1: Calibración de Cámara}
\label{chapter:introduction_ses_1}

\section{Materiales}
En esta práctica se trabajará con los siguientes recursos (puede encontrarlos en la sección de Moodle \textit{Laboratorio/Sesión 1}):

\begin{itemize}
    \item \textbf{lab\_1ipynb}: notebook con el código que deberá completar.
    \item \textbf{left}: carpeta con imágenes de la cámara situada a la izquierda.
    \item \textbf{right}: carpeta con imágenes de la cámara situada a la derecha.
    \item \textbf{fisheye}: carpeta con imágenes de una cámara con lente ojo de pez.
\end{itemize}

\section{Apartados de la práctica}
La Sesión 1 del laboratorio está dividida en los siguientes apartados:

\begin{itemize}
    \item Apartado A: Calibración de cámara (izquierda y derecha).
    \item Apartado B:  Corrección de distorsión (ojo de pez).
\end{itemize}

\section{Qué va a aprender}

Al finalizar esta práctica, sabrá cómo obtener los parámetros intrínsecos y extrínsecos de una cámara, obteniendo así un modelo. También aprenderá a aplicar un modelo de cámara en un problema de distorsión de imagen. Además, se habrá familiarizado con la librería OpenCV \footnote{\href{https://opencv.org/}{OpenCV}: https://opencv.org/} de Python.