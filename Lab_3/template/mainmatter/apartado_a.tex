\chapter{Apartado A: \textbf{Detección de esquinas}}
\label{chapter:tarea_a}


\section*{Tarea A.1: Creación de carpeta de resultados}
\phantomsection
\addcontentsline{toc}{section}{Tarea A.1: Creación de carpeta de resultados}

Cree una nueva capeta llamada \texttt{partA}, dentro de la carpeta \texttt{data}, con el objetivo de presentar en ella los resultados de esta parte de la práctica.

\section*{Tarea A.2: Carga de imágenes}
\addcontentsline{toc}{section}{Tarea A.2: Carga de imágenes}

Defina y ejecute los dos métodos propuestos para cargar imágenes \texttt{imageio\_load\_images()} y \texttt{opencv\_load\_images()}. Observe lo que ocurre al guardar ambas imágenes usando la misma función \texttt{cv2.imwrite()}.

\section*{Preguntas}
\addcontentsline{toc}{section}{Preguntas}

\vspace{5mm}
\begin{tcolorbox}[colback=gray!10, colframe=gray!30, coltitle=black, title=Pregunta A.1, halign=left]
Pregunta A.1
\end{tcolorbox}