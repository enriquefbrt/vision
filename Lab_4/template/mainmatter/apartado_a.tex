\chapter{Apartado A: \textbf{Sustracción de Fondo}}
\label{chapter:tarea_a}


\section*{Tarea A.1: Carga de vídeo}
\phantomsection
\addcontentsline{toc}{section}{Tarea A.1: Carga de vídeo}
Cargue un vídeo en el cual se detectarán objetos en movimiento. Puede utilizar un vídeo local o la cámara en tiempo real.

\section*{Tarea A.2: Sustracción de fondo mediante diferencia de frames}
\phantomsection
\addcontentsline{toc}{section}{Tarea A.2: Sustracción de fondo mediante diferencia de frames}
Realice una sustracción de fondo mediante diferencia de frames, para ello guarde un frame con el fondo estático y úselo como frame de referencia de fondo.

\section*{Tarea A.3: Configuración de la Sustracción de Fondo con GMM}
\phantomsection
\addcontentsline{toc}{section}{Tarea A.3: Configuración de la Sustracción de Fondo con GMM}

Configure el sustractor de fondo usando el modelo de mezcla de gaussianas adaptativas (MOG2).

\section*{Tarea A.4: Aplicación de la Sustracción de Fondo}
\phantomsection
\addcontentsline{toc}{section}{Tarea A.4: Aplicación de la Sustracción de Fondo}
Aplique la sustracción de fondo en cada frame para extraer los objetos en movimiento.

\section*{Tarea A.5: Visualización y Guardado de Resultados}
\phantomsection
\addcontentsline{toc}{section}{Tarea A.5: Visualización y Guardado de Resultados}
Visualice el resultado de la detección de movimiento y guarde la imagen binaria resultante en la carpeta \texttt{data/results}.

\section*{Preguntas}
\addcontentsline{toc}{section}{Preguntas}

\vspace{5mm}
\begin{tcolorbox}[colback=gray!10, colframe=gray!30, coltitle=black, title=Pregunta A.1, halign=left]
¿Cómo afecta la variable \texttt{varThreshold} a la precisión de la detección?
\end{tcolorbox}

\vspace{5mm}
\begin{tcolorbox}[colback=gray!10, colframe=gray!30, coltitle=black, title=Pregunta A.2, halign=left]
¿Qué ventajas presenta \texttt{createBackgroundSubtractorMOG2} frente a métodos simples de diferencia de imágenes?
\end{tcolorbox}