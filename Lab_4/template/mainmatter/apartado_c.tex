\chapter{Apartado C: \textbf{Filtro de Kalman para Seguimiento de Objetos}}
\label{chapter:tarea_c}

\section*{Tarea C.1: Configuración del Filtro de Kalman}
\phantomsection
\addcontentsline{toc}{section}{Tarea C.1: Configuración del Filtro de Kalman}
Inicialice el filtro de Kalman con una matriz de medición y transición adecuada para un seguimiento en dos dimensiones.

\section*{Tarea C.2: Predicción y Corrección del Estado}
\phantomsection
\addcontentsline{toc}{section}{Tarea C.2: Predicción y Corrección del Estado}
Realice la predicción del estado y corrija la posición estimada en cada iteración.

\section*{Preguntas}
\addcontentsline{toc}{section}{Preguntas}

\vspace{5mm}
\begin{tcolorbox}[colback=gray!10, colframe=gray!30, coltitle=black, title=Pregunta C.1, halign=left]
¿Cómo afecta el valor de \texttt{transitionMatrix} a la predicción en el filtro de Kalman?
\end{tcolorbox}

\vspace{5mm}
\begin{tcolorbox}[colback=gray!10, colframe=gray!30, coltitle=black, title=Pregunta C.2, halign=left]
¿Cuál es la diferencia entre \texttt{measurementMatrix} y \texttt{transitionMatrix} en el contexto del seguimiento de objetos?
\end{tcolorbox}