\chapter{\textbf{Primeros Pasos}}
\label{chapter:primeros_pasos}

\subsection*{Hardware}
\phantomsection
\addcontentsline{toc}{section}{Hardware}
\vspace{5mm}

Al comienzo de la sesión dispondrá de:

\begin{itemize}
    \item \textbf{Raspberry Pi}.
    \item \textbf{Camera module 3 WIDE (120º FOV)}.
    \item \textbf{Cable de alimentación a USB3}.
    \item \textbf{Cable HDMI to micro-HDMI (opcional)}
\end{itemize}

Realice el montaje según las indicaciones de su profesor. Todas las Rasspberry tienen flasheadas un sistema operativo compatible y con SSH activado. Aunque se recomienda el uso de un cable HDMI para facilitar el proceso, puede acceder a su Raspberry de múltiples formas:

Puede encontrar la IP address de su placa o su \textit{host name} para conectarse, o bien por SSH (SSH host-name.local) o por VNC (connect to host-name). Para conectarse por SSH puede utilizar Putty. Para conectarte por VNC puede usar RealVNC. No es necesario usar un programa concreto para la conexión con la Raspberry, use el que mejor se adecúe a sus necesidades. Regístrese con las credenciales de la Tabla \ref{table:credenciales}:

\begin{table}[h!]
    \centering
    \begin{tabular}{|l|l|}
    \hline
    \textbf{Campo} & \textbf{Valor}\\
    \hline
    Usuario & pi \\
    \hline
    Contraseña & imat@ICAI2024 \\
    \hline
    IP (Comillas) & 10.120.107.<número> \\
    \hline
    Nombre del equipo (hostname) en Comillas & imat-rpi-<xyz>.rpi-deac.alumnos.upcont.es \\
    \hline
    Nombre del equipo en otra red & imat-rpi-<xyz> \\
    \hline
    \end{tabular}
    \caption{Credenciales para iniciar sesión en la Raspberry Pi. Tenga en cuenta que <número> es el código de la placa que le han dado (por ejemplo 42), mientras que código <xyz> es el mismo número precedido por 0 (por ejemplo 042).}
    \label{table:credenciales}
\end{table}

Para comprobar que la cámara está conectada, ejecute el archivo que encontrará en la carpeta del projecto final: \texttt{test.py}. Como resultado, si se ha conectado directamente a un monitor, o si lo ha hecho con escritorio remoto, debería ver el vídeo captado por la cámara en tiempo real.

Si trabaja de manera cómoda implementando el código directamente en la Raspberry, puede optar por instalar Visual Studio Code desde el terminal de su Raspberry. Esta es una buena opción, puesto que podrá integrar el soporte Git de VSCode directamente. Utilice los siguientes comandos:\\

\texttt{sudo apt-get update}\\
\texttt{sudo apt-get install code}

\subsection*{Sesión Inicial}
\phantomsection
\addcontentsline{toc}{section}{Sesión Inicial}
\vspace{5mm}
En la primera sesión se espera que se aborden los siguientes puntos:

\begin{itemize}
    \item \textbf{Montaje}: Conexión con la Raspberry. Acceso con escritorio remoto, SSH o mediante cable HDMI.
    \item \textbf{Conexión} de cámara y recepción de vídeo.
    \item \textbf{Repositorio del proyecto}. Primer commit con el código de lectura de vídeo desde la cámara. Recuerde que se debe crear un ReadMe con la descripción del proyecto. Deberá hacer el proyecto público y compartir el enlace al mismo con el profesor.
    \item  \textbf{Planteamiento y diseño del proyecto}. Recuerde que antes de implementar su proyecto, debe contar con la aprobación del profesor.
\end{itemize}